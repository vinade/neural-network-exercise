\documentclass{article}
\usepackage[utf8]{inputenc}
\usepackage{amsmath}
\begin{document}



\centerline{\sc \large Função de ativação: Softmax}
\vspace{.5pc}
\centerline{\sc Resumo teórico}
\vspace{4pc}

Função:
\[ \pi_i = \frac{e^{x_i}}{\sum_{j=1}^{n} e^{x_j}} \]
\[ \frac{\partial \mathbf{\pi}}{\partial \mathbf{x}} = \frac{e^{\mathbf{x}}}{\sum_{j=1}^{n} e^{x_j}} \]
\vspace{3pc}

Como a função de probabilidade é uma função vetorial, a sua derivada é uma matriz jacobiana:

\begin{equation*}
\mathbf{J_{\pi}}(\mathbf{x}) = 
\begin{pmatrix}
\frac{\partial \pi_1}{\partial x_1} & \frac{\partial \pi_1}{\partial x_2} & \cdots & \frac{\partial \pi_1}{\partial x_n} \\
\frac{\partial \pi_2}{\partial x_1} & \frac{\partial \pi_2}{\partial x_2} & \cdots & \frac{\partial \pi_2}{\partial x_n} \\
\vdots  & \vdots  & \ddots & \vdots  \\
\frac{\partial \pi_m}{\partial x_1} & \frac{\partial \pi_m}{\partial x_2} & \cdots & \frac{\partial \pi_m}{\partial x_n}
\end{pmatrix}
\end{equation*}

Usando a derivada logarítmica:
\[ \frac{\partial \pi_i}{\partial x_j} = \pi_i \cdot \frac{\partial log(\pi_i)}{\partial x_j} \]
\[ log(\pi_i) = log(\frac{e^{x_i}}{\sum_{l=1}^{n} e^{x_l}}) = x_i - log(\sum_{l=1}^{n} e^{x_l})\]
\[ \frac{\partial log(\pi_i)}{\partial x_j} = \frac{\partial x_i}{\partial x_j} - \frac{\partial log(\sum_{l=1}^{n} e^{x_l})}{\partial x_j} \]
\[ \frac{\partial x_i}{\partial x_j} = \left\{\begin{array}{@{}l@{}}
    1 : i = j
    \\
    0 : i \neq j
  \end{array}\right. \]

\[ \frac{\partial log(\sum_{l=1}^{n} e^{x_l})}{\partial x_j} = \frac{e^{x_j}}{\sum_{l=1}^{n} e^{x_l}} \]
\[ \frac{\partial log(\pi_i)}{\partial x_j} = 1(i=j) - \frac{e^{x_j}}{\sum_{l=1}^{n} e^{x_l}} \]
\[ \frac{\partial log(\pi_i)}{\partial x_j} = 1(i=j) - \pi_j \]
\[ \frac{\partial \pi_i}{\partial x_j} = \pi_i \cdot \frac{\partial log(\pi_i)}{\partial x_j} = \pi_i \cdot (1(i=j) - \pi_j)  \]
\[ \frac{\partial \pi_i}{\partial x_j} = \pi_i(i=j) - \pi_i \cdot \pi_j  \]
\[ \frac{\partial \pi_i}{\partial x_j} = \left\{\begin{array}{@{}l@{}}
    \pi_i \cdot (1 - \pi_j) : i = j
    \\
    - \pi_i \cdot \pi_j : i \neq j
  \end{array}\right.  \]


Revisitando a matriz jacobiana:

\begin{equation*}
\mathbf{J_{\pi}}(\mathbf{x}) = 
\begin{pmatrix}
\pi_1 \cdot (1 - \pi_1) & - \pi_1 \cdot \pi_2 & \cdots & - \pi_1 \cdot \pi_n \\
- \pi_2 \cdot \pi_1 & \pi_2 \cdot (1 - \pi_2) & \cdots & - \pi_2 \cdot \pi_n \\
\vdots  & \vdots  & \ddots & \vdots  \\
- \pi_m \cdot \pi_1 & - \pi_m \cdot \pi_2 & \cdots & \pi_m \cdot (1 - \pi_n)
\end{pmatrix}
\end{equation*}

Para a implementação é possível criar a jacobiana usando uma matrix identidade:
\[ \mathbf{J_{\pi}}(\mathbf{x}) = \pi \cdot \mathbf{I} - \pi^{T} \cdot \pi \]

\end{document}